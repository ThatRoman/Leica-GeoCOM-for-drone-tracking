\hypertarget{development-of-a-guidance-and-automatic-positioning-system-for-builder-drones-with-a-total-station}{%
\section{Development of a guidance and automatic positioning system for
builder drones with a total
station}\label{development-of-a-guidance-and-automatic-positioning-system-for-builder-drones-with-a-total-station}}

\hypertarget{credit}{%
\subsection{Credit}\label{credit}}

Authors : \href{https://github.com/nicolassorensen/}{Nicolas Sorensen}
\& \href{https://github.com/rvermeiren/}{Rémy Vermeiren}

Promotor :
\href{https://uclouvain.be/fr/repertoires/pierre.latteur}{Pierre
Latteur} \& \href{https://uclouvain.be/fr/repertoires/ramin.sadre}{Ramin
Sadre}

Assistant :
\href{https://uclouvain.be/fr/repertoires/sebastien.goessens}{Sebastien
Goessens}

\hypertarget{repository}{%
\subsubsection{Repository :}\label{repository}}

Based on the work of
\href{https://github.com/art-mx/leica_ros_sph}{Maxim Artyom} and
\href{https://github.com/georgwi/leica_ros_sph}{Georg Wiedebach}

\href{https://trello.com/b/cHMLdS54/m\%C3\%A9moire}{Trello} and
\href{https://tfebuildingwithdrones.slack.com/}{Slack}

\hypertarget{resources}{%
\subsubsection{Resources}\label{resources}}

\href{https://github.com/adam-p/markdown-here/wiki/Markdown-Cheatsheet}{Markdown
Cheatsheet}

\href{https://www.python.org/dev/peps/pep-0258/}{Python documentation
convention}

\hypertarget{installation}{%
\subsection{Installation}\label{installation}}

\hypertarget{requirements}{%
\subsubsection{Requirements}\label{requirements}}

\begin{itemize}
\tightlist
\item
  Windows 10
\item
  Leica cable GEV267 for Total station TCRP1203
\item
  Leica cable GEV269 for Total station MS50, MS60
\item
  Total station with GeoCom support (TCRP1203, MS50, MS60)
\item
  \href{https://www.python.org/download/releases/2.7/}{Python2.7}
\item
  pip package manager for Python 2.7 (normally already include in Python
  installation)
\item
  \href{https://pypi.python.org/pypi/pyserial/2.7}{pyserial}

  \begin{itemize}
  \tightlist
  \item
    Install with
    \texttt{C\textbackslash{}:Python27\textbackslash{}python.exe\ -m\ pip\ install\ pyserial}
    in command line
  \end{itemize}
\item
  USB cable drivers for GEV267,GEV268,GEV269 V3.0 available on Leica
  MyWorld website

  \begin{itemize}
  \tightlist
  \item
    Install the driver :

    \begin{itemize}
    \tightlist
    \item
      Download the driver and extract it
    \item
      Plug-in the cable on the Windows computer
    \item
      Go to ``Device Manager''
    \item
      Find the device ``FT232R''
    \item
      Right click and select ``Update Driver''
    \item
      Browse your computer into the extracted folder
    \item
      Select the folder ``Windows XP, Server 2003, Server 2008 R2,Vista,
      7, 8''
    \item
      Click on next to finish the installation
    \end{itemize}
  \item
    After this process, it's possible you need to redo this with the new
    device showed in the list call ``Serial USB Converter''
  \end{itemize}
\end{itemize}

\hypertarget{get-the-source-codes}{%
\subsubsection{Get the source codes}\label{get-the-source-codes}}

You can download the code on this repository:
https://github.com/rvermeiren/Leica-GeoCOM-for-drone-tracking If the
link is dead, contact \href{https://github.com/nicolassorensen/}{Nicolas
Sorensen} or \href{https://github.com/rvermeiren/}{Rémy Vermeiren}

\#\#Run and Usage \#\#\# Run

\begin{verbatim}
$ C:\Python27\python.exe alx_track.py
-d (verbose for debug)
-b (big prism -- default = mini-prism)
-p "port" (ex: "COM1" -- This can be found in "Device Manager on Windows")
\end{verbatim}

\hypertarget{usage}{%
\subsubsection{Usage}\label{usage}}

Use \texttt{-h} to show usage

\begin{verbatim}
  Options:
  -h,           --help            show this help message and exit
  -p PORT,      --port=PORT       specify used port [default: /dev/ttyUSB0]
  -b BAUDRATE,  --baudrate=BAUDRATE
                                  specify used baudrate [default: "COM3"]
  -d, --debug                     show debug information
  -B, --Big                       use big 360 prism
\end{verbatim}

\hypertarget{licenses}{%
\subsection{Licenses}\label{licenses}}

The original work was under the following copyright :

\begin{verbatim}
#Copyright (c) 2013, Marcel Schoch, ASL, ETH Zurich, Switzerland
#You can contact the author at <slynen at ethz dot ch>
#
#All rights reserved.
#
#Redistribution and use in source and binary forms, with or without
#modification, are permitted provided that the following conditions are met:
# * Redistributions of source code must retain the above copyright
#notice, this list of conditions and the following disclaimer.
# * Redistributions in binary form must reproduce the above copyright
#notice, this list of conditions and the following disclaimer in the
#documentation and/or other materials provided with the distribution.
# * Neither the name of ETHZ-ASL nor the
#names of its contributors may be used to endorse or promote products
#derived from this software without specific prior written permission.
#
#THIS SOFTWARE IS PROVIDED BY THE COPYRIGHT HOLDERS AND CONTRIBUTORS "AS IS" AND
#ANY EXPRESS OR IMPLIED WARRANTIES, INCLUDING, BUT NOT LIMITED TO, THE IMPLIED
#WARRANTIES OF MERCHANTABILITY AND FITNESS FOR A PARTICULAR PURPOSE ARE
#DISCLAIMED. IN NO EVENT SHALL ETHZ-ASL BE LIABLE FOR ANY
#DIRECT, INDIRECT, INCIDENTAL, SPECIAL, EXEMPLARY, OR CONSEQUENTIAL DAMAGES
#(INCLUDING, BUT NOT LIMITED TO, PROCUREMENT OF SUBSTITUTE GOODS OR SERVICES;
#LOSS OF USE, DATA, OR PROFITS; OR BUSINESS INTERRUPTION) HOWEVER CAUSED AND
#ON ANY THEORY OF LIABILITY, WHETHER IN CONTRACT, STRICT LIABILITY, OR TORT
#(INCLUDING NEGLIGENCE OR OTHERWISE) ARISING IN ANY WAY OUT OF THE USE OF THIS
#SOFTWARE, EVEN IF ADVISED OF THE POSSIBILITY OF SUCH DAMAGE.
\end{verbatim}
